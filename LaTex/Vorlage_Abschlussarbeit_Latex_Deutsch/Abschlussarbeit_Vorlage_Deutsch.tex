\documentclass[a4paper,12pt]{article}
%   Standards
    \usepackage[latin1]{inputenc}
    \usepackage{fontenc}
%   Bibliography style
%    \usepackage{chicago}

% German language support
  % \usepackage{german}
  % \usepackage{bibgerm}
	 \usepackage[ngerman]{babel}

%   Mathematics
    \usepackage{amsmath}
    \usepackage{amssymb}
    \usepackage{amsthm}
    \usepackage{amscd}
    \usepackage{amsfonts}

%   Formatting
    \usepackage{graphicx}   %Graphics
    \usepackage{booktabs}   %Tables
    \usepackage{a4}         %Page setting
    \usepackage{fancyhdr}   %Headers and footers
    \usepackage{longtable}  %Tabellen l\"{a}nger als eine Seite
    \usepackage{subfigure}
    \usepackage{enumerate}
    \usepackage{epsfig}
    \usepackage{layout}
    \usepackage{tabularx}
    \usepackage{array}
    \usepackage{wasysym}
    \usepackage{fancybox}
    \usepackage{color}
    \usepackage{rotating}
    \usepackage{slashbox}% kann in Tabelle diagonalen Strich darstellen
    \usepackage{multirow}
    \usepackage{natbib}		% Anpassung
		%\usepackage{jurabib}

    %Calculation: 1Inch = 2.54 cm
        \setlength{\oddsidemargin}{0.5cm}
        \setlength{\textwidth}{15,5cm}
        \setlength{\textheight}{23cm}

    %Chapter pages
        \pagestyle{headings}
%        \pagestyle{fancy}
%        \lhead{\emph{COLLATERALIZED DEBT OBLIGATIONS}}
%        \chead{}
%        \rhead{\thepage}
%        \lfoot{}
%        \cfoot{}
%        \rfoot{}
%        \renewcommand{\headrulewidth}{0pt}
%        \renewcommand{\footrulewidth}{0pt}

    % Chapter title pages
        \fancypagestyle{plain}{
        \fancyhf{}
        \fancyhead[L]{\emph{Titel der Arbeit}}
        \fancyhead[R]{\thepage}
        \fancyfoot[L]{}
        \fancyfoot[R]{}
        \renewcommand{\headrulewidth}{0pt}
        \renewcommand{\footrulewidth}{0pt}}
        \renewcommand{\baselinestretch}{1.3}
				\let\footnoteOld\footnote		% Zeilenabstand in der Fu�note wird zur�ckgesetzt
				\renewcommand{\footnote}[1]{\linespread{1.0}\footnoteOld{#1}\linespread{1.2}}		% Zeilenabstand in der Fu�note wird gesetzt

\begin{document}

\begin{titlepage}
\topskip0cm
\begin{center}
{\Large Karlsruher Institut f�r Technologie\\[0.4cm]
Institut f\"{u}r Finanzwirtschaft, Banken und Versicherungen\\[0.3cm]
Lehrstuhl Financial Engineering und Derivate\\[0.3cm]
Prof. Dr. Marliese Uhrig-Homburg}\\[3.5cm]
{\large Masterarbeit}\\[1.5cm]
{\Huge Titel des Themas}\\[8cm]
\end{center}
\renewcommand{\baselinestretch}{1.2}\small\normalsize
\begin{tabular}{ll}
        Verfasser:  & Vorname Name\\
                    & Stra{\ss}e + Hausnr\\
                    & PLZ Ort\\
										& E-Mail: e-mail Adresse\\\\
        \multicolumn{2}{l}{Karlsruhe, den xx. September 20xx}
    \end{tabular}
    \vfill
\end{titlepage}

\setcounter{page}{1}\renewcommand{\thepage}{\roman{page}}%
\tableofcontents %
\newpage
% \listoffigures \addcontentsline{toc}{chapter}{List of Figures}% in Englisch
\listoffigures \addcontentsline{toc}{section}{Abbildungsverzeichnis} % in Deutsch
% \listoftables \addcontentsline{toc}{chapter}{List of Tables}% in Englisch
\newpage
\listoftables \addcontentsline{toc}{section}{Tabellenverzeichnis} % in Deutsch
\newpage
\setcounter{page}{1}\renewcommand{\thepage}{\arabic{page}}

\section{Einleitung\label{sec:1}}

Nach \citet{Hull2000} gilt...  % natbib verwendet citet statt cite

\subsection{Motivation\label{sec:1.1}}

% hier folgt die Motivation

%hier die richtige Gestalt der Fu{\ss}note \footnote{Siehe Burghof et al (2000), Kapitel I.1.1 \emph{Kreditrisiko - Definition und Systematisierung}, p 3-6; and
%Sch\"{o}nbucher (2003), Kapitel I.2 \emph{The Components of Credit
%Risk}, p 2f.}

\section{\"{U}berschrift Kapitel 2\label{sec:2}}

Hier folgt Kapitel 2.\footnote{Vgl. \citet{Cox1979}, S. 230.}

\subsection{"Uberschrift Unterarbschnitt 2.1\label{sec:2.1}}

Hier folgt das erste Unterkapitel.\footnote{Vgl. \citet{Skantze2000}, S. 22.}

\subsection{"Uberschrift Unterarbschnitt 2.2\label{sec:2.2}}

%Einbinden von einer Abbildungen
%\begin{figure}[htbp]
%\begin{center}
%  \includegraphics[width=0.80\textwidth]{Bilder/name_bild.eps}\\
%  \caption[ggf. abweichender Eintrag f\"{u}r Abbildungsverzeichnis]{Beschreibung der Graphik}
%\end{center}
%\end{figure}

% Einbinden von mehreren Bildern nebeneinander
%
%\begin{figure}[h]
%\centering
%        \subfigure[Unterschrift Bild1.]{\label{Ref_Bild1}\includegraphics*[width=.45\textwidth]{Bilder/Name1.eps}}\quad
%        \subfigure[Unterschrift Bild2.]{\label{Ref_Bild2}\includegraphics*[width=.45\textwidth]{Bilder/Name2.eps}}\quad
%        \caption[Eintrag f\"{u}r Abbildungsverzeichnis]{Unterschrift des Bildes}\label{Referenz Gesamtabbildung}
%\end{figure}

%% Apendix
%\addcontentsline{toc}{section}{Bibliography} \nocite{*}
\newpage
\addcontentsline{toc}{section}{\bibname} \nocite{*} % Deutsch
\bibliographystyle{apalike_ger_manipulation}
\bibliography{Abschluss_Bib}
\newpage
\begin{appendix}
\section{\"{U}berschift Anhang A}

% Text + weiterer Inhalt

\end{appendix}
\newpage \thispagestyle{empty}

\begin{center}
\section*{Erkl\"{a}rung}
\end{center}
%\thispagestyle{empty}%

\vspace{2cm}
\begin{flushleft}
Ich erkl"are hiermit ehrenw"ortlich, dass ich die vorliegende Masterarbeit\\[-0.3cm]
\end{flushleft}
\begin{center}
{\large Titel der Arbeit}\\[0.5cm]
\end{center}
selbst\"{a}ndig angefertigt habe. Die aus fremden Quellen direkt oder indirekt
"ubernommenen Gedanken sind als solche kenntlich gemacht.\\[2.5cm]

\begin{flushleft}
Ort, den 1. Juli 2011\\[0.1cm]
\end{flushleft}
\hspace*{9.0cm}.....................................................\\
\hspace*{10.1cm}(Vorname Name)
%\rightline{Muster Mustermann\hspace{4cm}}\\


\end{document}
